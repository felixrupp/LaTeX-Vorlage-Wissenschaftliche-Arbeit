% !TEX encoding = UTF-8 Unicode

% ------------------------------------------------------------------------------------------------------
%	Formatvorlage für wissenschaftliche Arbeiten (Diplomarbeit, Bachelorarbeit, Masterarbeit)
% ------------------------------------------------------------------------------------------------------
%	ursprünglich erstellt von Stefan Macke, 24.04.2009
%	http://blog.stefan-macke.de
%
%	erweitert von Felix Rupp
%	http://www.felixrupp.com/
%
%	Version: 1.2
%	Datum: 21.05.2013


% Dokumentenkopf ---------------------------------------------------------------------------------------
%   Diese Vorlage basiert auf "scrreprt" aus dem koma-script.
% ------------------------------------------------------------------------------------------------------
\documentclass[
    12pt, % Schriftgröße
    DIV10, % Änderung der Größe des Satzspiegels (bedruckbarer Bereich einer Seite), nur in Verbindung mit koma-script verwendbar
    ngerman, % für Umlaute, Silbentrennung etc.
    a4paper, % Papierformat
    oneside, % einseitiges Dokument
    titlepage, % es wird eine Titelseite verwendet
    parskip=half, % Abstand zwischen Absätzen (halbe Zeile)
    headings=normal, % Größe der Überschriften verkleinern
    listof=totoc, % Verzeichnisse im Inhaltsverzeichnis aufführen
    bibliography=totoc, % Literaturverzeichnis im Inhaltsverzeichnis aufführen
    index=totoc, % Index im Inhaltsverzeichnis aufführen
    captions=tableheading, % Beschriftung von Tabellen unterhalb ausgeben
    final % Status des Dokuments (final/draft)
]{scrreprt}

% UTF8 und T1 Fontencoding -----------------------------------------------------------------------------
\usepackage[utf8]{inputenc}
\usepackage[T1]{fontenc}


% Meta-Informationen -----------------------------------------------------------------------------------
%   Informationen über das Dokument, wie z.B. Titel, Autor, Matrikelnr. etc
%   werden in der Datei Meta.tex definiert und können danach global
%   verwendet werden.
% ------------------------------------------------------------------------------------------------------
% !TEX encoding = UTF-8 Unicode
% !TEX root =  Bachelorarbeit.tex

% Meta-Informationen ------------------------------------------------------------------------------------
%   Definition von globalen Parametern, die im gesamten Dokument verwendet
%   werden können (z.B auf dem Deckblatt etc.).
%
%   ACHTUNG: Wenn die Texte Umlaute oder ein Esszet enthalten, muss der folgende
%            Befehl bereits an dieser Stelle aktiviert werden:
%            \usepackage[latin1]{inputenc}
% -------------------------------------------------------------------------------------------------------
\newcommand{\titel}{Entwicklung einer TYPO3 Extension auf Basis von Extbase und Fluid}
\newcommand{\untertitel}{zur Live-Anzeige von Titelinformationen eines Webradios}
\newcommand{\untertitelDeckblatt}{zur Live-Anzeige von Titelinformationen\\ eines Webradios}
\newcommand{\art}{Bachelor-Thesis}
\newcommand{\fachgebiet}{zur Erlangung des akademischen Grades\\ Bachelor of Science (B.\,Sc.) im Studienfach\xspace}
\newcommand{\autor}{Felix Rupp}
\newcommand{\keywords}{Bachelorarbeit, Felix Rupp}
\newcommand{\studienbereich}{Medieninformatik\xspace}
\newcommand{\matrikelnr}{1234567}
\newcommand{\erstgutachter}{Prof. Dr. Max Mustermann}
\newcommand{\zweitgutachter}{Dipl.-Ing. (FH) Herbert Beispiel}
\newcommand{\jahr}{2011}
\newcommand{\hochschule}{Technischen Hochschule Mittelhessen}
\newcommand{\ort}{Friedberg}
\newcommand{\logo}{LogoMuster.pdf}
\newcommand{\creator}{TeXShop 3.16}



% benötigte Packages -----------------------------------------------------------------------------------
%   LaTeX-Packages, die benötigt werden, sind in die Datei Packages.tex
%   "ausgelagert", um diese Vorlage möglichst übersichtlich zu halten.
% ------------------------------------------------------------------------------------------------------
% !TEX encoding = UTF-8 Unicode
% !TEX root =  Bachelorarbeit.tex

% Anpassung des Seitenlayouts ---------------------------------------------------
%   siehe Seitenstil.tex
% -----------------------------------------------------------------------------------------
\usepackage[
    automark, % Kapitelangaben in Kopfzeile automatisch erstellen
    headsepline, % Trennlinie unter Kopfzeile
    ilines % Trennlinie linksbündig ausrichten
]{scrpage2}


% Anpassung an Landessprache -------------------------------------------------
\usepackage[ngerman]{babel}


% Umlaute ------------------------------------------------------------------------------
%   Umlaute/Sonderzeichen wie äüöß direkt im Quelltext verwenden (CodePage).
%   Erlaubt automatische Trennung von Worten mit Umlauten.
% -----------------------------------------------------------------------------------------
\usepackage[utf8]{inputenc}
\usepackage[T1]{fontenc}
\usepackage{textcomp} % Euro-Zeichen etc.


% Schrift --------------------------------------------------------------------------------
\usepackage{lmodern} % bessere Fonts
\usepackage{relsize} % Schriftgröße relativ festlegen


% Bessere Unterstreichungen ---------------------------------------------
\usepackage[normalem]{ulem}


% Grafiken -----------------------------------------------------------------------------
% Einbinden von JPG-Grafiken ermöglichen
\usepackage[dvips,final]{graphicx}
% hier liegen die Bilder des Dokuments
\graphicspath{{Bilder/}}


% Befehle aus AMSTeX für mathematische Symbole z.B. \boldsymbol \mathbb
\usepackage{amsmath,amsfonts}


% Eurozeichen benutzen ------------------------------------------------------------
\usepackage{eurosym}


% für Index-Ausgabe mit \printindex -----------------------------------------------
\usepackage{makeidx}


% Einfache Definition der Zeilenabstände und Seitenränder etc. ------------
\usepackage{setspace}
\usepackage{geometry}


% Symbolverzeichnis ------------------------------------------------------------------
%
%   Symbolverzeichnisse bequem erstellen. Beruht auf MakeIndex:
%     makeindex.exe %Name%.nlo -s nomencl.ist -o %Name%.nls
%   erzeugt dann das Verzeichnis unter WIndows. Dieser Befehl kann z.B. im TeXnicCenter
%   als Postprozessor eingetragen werden, damit er nicht ständig manuell
%   ausgeführt werden muss.
%
%   Unter Unix bitte einfach das mitgelieferte Shellskript "makemyindex" nutzen.
%
%   Die Definitionen werden im Fließtext mit den Befehlen:
%               - \Fachbegriff
%               - \FachbegriffSpezial
%               - \FachbegriffSpezialB
%    erzeugt oder separat in die Datei "Glossar.tex" eingetragen.
% -------------------------------------------------------------------------------------------
\usepackage[intoc]{nomencl}
\let\abbrev\nomenclature
\renewcommand{\nomname}{Abkürzungsverzeichnis und Glossar}
\setlength{\nomlabelwidth}{.25\hsize}
\renewcommand{\nomlabel}[1]{#1 \dotfill}
\setlength{\nomitemsep}{-\parsep}


% zum Umfließen von Bildern ---------------------------------------------------------
\usepackage[vflt]{floatflt}
\usepackage{subfigure}

% zum Einbinden von Programmcode -----------------------------------------------
\usepackage{listings}
\usepackage{xcolor} 
\definecolor{hellgelb}{rgb}{1,1,0.9}
\definecolor{colKeys}{rgb}{0.8,0,0.5}
\definecolor{colIdentifier}{rgb}{0.6,0,0.3}
\definecolor{colComments}{rgb}{0,0.5,0}
\definecolor{colString}{rgb}{0,0,1}

\lstset{
    float=htbp,
    basicstyle=\ttfamily\color{black}\small\smaller,
    identifierstyle=,%\color{colIdentifier},
    keywordstyle=\color{colKeys}\bfseries,
    stringstyle=\color{colString},
    commentstyle=\color{colComments},
    columns=flexible,
    tabsize=4,
    frame=single,
    extendedchars=true,
    showspaces=false,
    showstringspaces=false,
    numbers=left,
    numberstyle=\tiny,
    breaklines=true,
    backgroundcolor=\color{hellgelb},
    breakautoindent=true,
    escapeinside={(*}{*)},
    literate={Ö}{{\"O}}1 {Ä}{{\"A}}1 {Ü}{{\"U}}1 {ß}{{\ss}}2 {ü}{{\"u}}1 {ä}{{\"a}}1 {ö}{{\"o}}1 {µ}{\textmu}1
 }

% --------------------------------------------------------------------------------------------
% 
% Eigene Definitionen für Quelltext-Stile
%
% Define CSS:
\lstdefinelanguage{CSS}{
    morestring=[b]',
    morestring=[b]",
    comment=[l]{/*}{*/},
    sensitive=false,
    morekeywords={accelerator,adjust,after,align,attachment,azimuth,background,before,behavior,binding,border,bottom,bottomright,bottomleft,break,caption-side,char,clear,clip,color,colors,cue,cursor,collapse,decoration,direction,display,elevation,empty-cells,family,filter,float,flow,focus,font,grid,height,image,increment,indent,input,inside,ime-mode,include-source,justify,last,layer,left,letter,line,list,margin,marker,marks,max,min,mode,modify,moz,offsett,opacity,orphans,outline,overflow,overflow-X,overflow-Y,overhang,position,position-x,position-y,padding,page,pause,pitch,play-during,position,quotes,radius,range,repeat,replace,reset,richness,right,ruby,set-link-source,select,shadow,size,spacing,speak,speak-header,speak-numeral,speak-punctuation,speech-rate,stress,stretch,style,table-layout,text,transform,text-autospace,text-kashida-space,top,topleft,topright,type,underline,unicode-bidi,use-link-source,user,variant,vertical,visibility,voice-family,volume,white-space,weight,widows,width,word,wrap,word-wrap,writing-mode,z-index,zoom}
}


% Define JavaScript:
\lstdefinelanguage{JavaScript}{
  keywords={typeof, new, true, false, catch, function, return, null, catch, switch, var, if, in, while, do, else, case, break},
  keywordstyle=\color{colKeys}\bfseries,
  ndkeywords={class, export, boolean, throw, implements, import, this},
  ndkeywordstyle=\color{darkgray}\bfseries,
  sensitive=false,
  comment=[l]{//},
  morecomment=[s]{/*}{*/},
  morestring=[b]',
  morestring=[b]"
}

% Define TypoScript:
\lstdefinelanguage{TypoScript}{
  keywords=[1]{PAGE, HTML, TEXT, COA, COA\_INT, FILE, IMAGE, IMG\_RESOURCE, CLEARGIF, CONTENT, RECORDS, HMENU, CTABLE, OTABLE, COLUMNS, HRULER, IMGTEXT, CASE, LOAD\_REGISTER, RESTORE\_REGISTER, FORM, SEARCHRESULT, USER, USER\_INT, TEMPLATE, FLUIDTEMPLATE, MULTIMEDIA, SVG, EDITPANEL, GIFBUILDER, GMENU, TMENU, TMENUITEM, IMGMENU, IMGMENUITEM, JSMENU, JSMENUITEM, BOX},
  keywordstyle=[1]{\color{blue}\bfseries},
  keywords=[2]{plugin, view, page, file, text, config},
  keywordstyle=[2]{\color{blue}\bfseries},
  keywords=[3]{EXT},
  keywordstyle=[3]{\color{blue}\bfseries},
  sensitive=true,
  comment=[l]{\#\ }
}

% PHP5 dialect 
\lstdefinelanguage[5]{PHP}[]{PHP} {
  morekeywords={
  %--- class and exceptions keywords 
  class,static,private,public,abstract,interface,const,function,require_once,final,new,extends,implements,
  %--- additional array functions 
  array_combine,array_diff_uassoc,array_udiff,array_udiff_assoc,% 
  array_udiff_uassoc,array_walk_recursive,array_uintersect_assoc,% 
  array_usintersect_uassoc,array_uintersect,% 
  %--- string functions 
  str_split, strpbrk,substr_compare 
  %--- Date and time functions 
  idate,date_sunset,date_sunrise,time_nanosleep,
  %--- return
  return
  } 
} 


% Fluid dialect 
\lstdefinelanguage[Fluid]{HTML}[]{HTML} {
  morekeywords={if, for, each, as, condition, controller, arguments, image, link, action, class} 
} 


% URL verlinken, lange URLs umbrechen etc. -------------------------------------
\usepackage{url}


% natbib einbinden -----------------------------------------------------------------------
\usepackage[square,numbers,sort&compress]{natbib}


% PDF-Optionen -------------------------------------------------------------------------
\usepackage[
    bookmarks,
    bookmarksopen=true,
    colorlinks=true,
% diese Farbdefinitionen zeichnen Links im PDF farblich aus
    linkcolor=red, % einfache interne Verknüpfungen
    anchorcolor=black,% Ankertext
    citecolor=blue, % Verweise auf Literaturverzeichniseinträge im Text
    filecolor=magenta, % Verknüpfungen, die lokale Dateien öffnen
    menucolor=red, % Acrobat-Menüpunkte
    urlcolor=cyan, 
%
% diese Farbdefinitionen sollten für den Druck verwendet werden (alles schwarz):
%
    %linkcolor=black, % einfache interne Verknüpfungen
    %anchorcolor=black, % Ankertext
    %citecolor=black, % Verweise auf Literaturverzeichniseinträge im Text
    %filecolor=black, % Verknüpfungen, die lokale Dateien öffnen
    %menucolor=black, % Acrobat-Menüpunkte
    %urlcolor=black, 
%
    backref,
    plainpages=false, % zur korrekten Erstellung der Bookmarks
    pdfpagelabels, % zur korrekten Erstellung der Bookmarks
    hypertexnames=false, % zur korrekten Erstellung der Bookmarks
    linktocpage % Seitenzahlen anstatt Text im Inhaltsverzeichnis verlinken
]{hyperref}
% Befehle, die Umlaute ausgeben, führen zu Fehlern, wenn sie hyperref als Optionen übergeben werden
\hypersetup{
    pdftitle={\titel \untertitel},
    pdfauthor={\autor},
    pdfcreator={\creator},
    pdfsubject={\titel \untertitel},
    pdfkeywords={\keywords},
}

% Paket zum sauberen Einbauen von externen PDF-Dateien -----------------
\usepackage[final]{pdfpages}


% fortlaufendes Durchnummerieren der Fußnoten -------------------------------
\usepackage{chngcntr}


% schönere Tabellen --------------------------------------------------------------------
\usepackage{tabularx}
\usepackage{multirow}


% für lange Tabellen ---------------------------------------------------------------------
\usepackage{longtable}
\usepackage{ltxtable}
\usepackage{filecontents}
\usepackage{array}
\usepackage{ragged2e}
\usepackage{lscape}


% Rotation von Elementen -------------------------------------------------------
\usepackage{rotating}


% Formatierung von Listen ändern --------------------------------------------------
\usepackage{paralist}
\usepackage{enumitem}


% bei der Definition eigener Befehle benötigt -------------------------------------
\usepackage{ifthen}
\usepackage{forloop}


% definiert u.a. die Befehle \todo und \listoftodos --------------------------------
\usepackage{todonotes}


% sorgt dafür, dass Leerzeichen hinter parameterlosen Makros nicht als Makroendezeichen interpretiert werden
\usepackage{xspace}



% Erstellung eines Index und Abkürzungsverzeichnisses/Glossars aktivieren ------------------------------
\makeindex{}
\makenomenclature{}


% Kopf- und Fußzeilen, Seitenränder etc. ---------------------------------------------------------------
% !TEX encoding = UTF-8 Unicode
% !TEX root =  Bachelorarbeit.tex

% Zeilenabstand 1,5 Zeilen ---------------------------------------------------------------------
\onehalfspacing{}


% Seitenränder ------------------------------------------------------------------------------------
\setlength{\topskip}{\ht\strutbox} % behebt Warnung von geometry
\geometry{paper=a4paper,left=35mm,right=35mm,top=35mm}


% Kopf- und Fußzeilen --------------------------------------------------------------------------
\pagestyle{scrheadings}

% Kopf- und Fußzeile auch auf Kapitelanfangsseiten
\renewcommand*{\chapterpagestyle}{scrheadings} 

% Schriftform der Kopfzeile
\renewcommand{\headfont}{\normalfont}


% Kopfzeile für einseitiges Dokument:
\ihead{%\small{\textsc{\titel}	\\
	%\untertitel
	%\\[2ex]
	\textit{\headmark}%}
}
\chead{}
\rohead{\includegraphics[scale=0.06]{\logo}}
\setlength{\headheight}{21mm} % Höhe der Kopfzeile
% Kopfzeile über den Text hinaus verbreitern
\setheadwidth[0pt]{textwithmarginpar} 
\setheadsepline[text]{0.4pt} % Trennlinie unter Kopfzeile


%% Kopfzeile für Zweiseitiges Dokument:
%\ihead{\textit{\leftmark}}
%\chead{}
%\ohead{}
%\lehead{\includegraphics[scale=0.25]{\logo}}
%\rohead{\includegraphics[scale=0.25]{\logo}}
%\setheadsepline[text]{0.4pt} % Trennlinie unter Kopfzeile


% Fußzeile
%\ifoot{\copyright\ \autor}
\cfoot{}
\ofoot{\pagemark \\[4ex]}


% sonstige typographische Einstellungen ---------------------------------------------------
% erzeugt ein wenig mehr Platz hinter einem Punkt
\frenchspacing{}

% Schusterjungen und Hurenkinder vermeiden
\clubpenalty = 10000
\widowpenalty = 10000 
\displaywidowpenalty = 10000

% Quellcode-Ausgabe formatieren
\lstset{numbers=left, numberstyle=\tiny, numbersep=5pt, breaklines=true}
\lstset{emph={square}, emphstyle=\color{red}, emph={[2]root,base}, emphstyle={[2]\color{blue}}}

% Fußnoten fortlaufend durchnummerieren
\counterwithout{footnote}{chapter}

% Biblatex Einstellungen -------------------------------------------------------------------------------
% Anpassung des Verzeichnisses
\setlength\bibitemsep{2\itemsep} % Verdoppeln des Abstandes zwischen den Einträgen
\setlength{\bibhang}{0.2cm} % Einzug nach erster Zeile eines Eintrags

% Fette Labels der Einträge
\xpretobibmacro{author}{\mkbibbold\bgroup}{}{}
\xapptobibmacro{author}{\egroup}{}{}
\xpretobibmacro{bbx:editor}{\mkbibbold\bgroup}{}{}
\xapptobibmacro{bbx:editor}{\egroup}{}{}
\renewcommand*{\labelnamepunct}{\mkbibbold{\addcolon\space}}




% eigene Definitionen für Silbentrennung ---------------------------------------------------------------
% !TEX encoding = UTF-8 Unicode
% !TEX root =  Bachelorarbeit.tex

% Trennvorschläge im Text werden mit \" angegeben
% untrennbare Wörter und Ausnahmen von der normalen Trennung können in dieser
% Datei mittels \hyphenation definiert werden

\hyphenation{End-an-wen-der}
\hyphenation{sym-bo-li-siert}
\hyphenation{for-ma-te}
\hyphenation{Ma-nage-ment}
\hyphenation{An-for-de-rungs-be-schrei-bung}
\hyphenation{Cor-po-rate}
\hyphenation{De-sign}
\hyphenation{Pro-gram-mier-schnitt-stel-le}


% eigene LaTeX-Befehle ---------------------------------------------------------------------------------
% !TEX encoding = UTF-8 Unicode
% !TEX root =  Bachelorarbeit.tex
% Eigene Befehle und typographische Auszeichnungen für diese


% einfaches Wechseln der Schrift, z.B.: \changefont{cmss}{sbc}{n} ---------------------------------------
\newcommand{\changefont}[3]{\fontfamily{#1} \fontseries{#2} \fontshape{#3} \selectfont}


% Abkürzungen mit korrektem Leerraum --------------------------------------------------------------------
\newcommand{\ua}{\mbox{u.\,a.\ }}
\newcommand{\zB}{\mbox{z.\,B.\ }}
\newcommand{\dahe}{\mbox{d.\,h.\ }}
\newcommand{\Vgl}{Vgl.\ }
\newcommand{\bzw}{bzw.\ }
\newcommand{\evtl}{evtl.\ }

\newcommand{\Abbildung}[1]{Abbildung~\ref{fig:#1}}

\newcommand{\bs}{$\backslash$}


% erzeugt ein Listenelement mit fetter Überschrift ------------------------------------------------------
\newcommand{\itemd}[2]{\item{\textbf{#1}}\\{#2}}


% einige Befehle zum Zitieren ---------------------------------------------------------------------------
\newcommand{\Zitat}[2][\empty]{\ifthenelse{\equal{#1}{\empty}}{\citep{#2}}{\citep[#1]{#2}}}
\newcommand{\ZitatSpezialA}[3]{\citep[#1][#2]{#3}}
\newcommand{\ZitatSpezialB}[2]{\citep[#1][]{#2}}


% zum Ausgeben von Autoren
\newcommand{\AutorName}[1]{\textsc{#1}}
\newcommand{\Autor}[1]{\AutorName{\citeauthor{#1}}}


% verschiedene Befehle um Wörter semantisch auszuzeichnen -----------------------------------------------
\newcommand{\NeuerBegriff}[1]{\textbf{#1}}

\newcommand{\Fachbegriff}[2][\empty]{\ifthenelse{\equal{#1}{\empty}}{\textit{#2}\xspace}{\textit{#2}\xspace\footnote{#1}\nomenclature{#2}{#1}}}

\newcommand{\FachbegriffSpezialA}[4]{\textit{#4}\footnote{#3}\label{fn:#1}\nomenclature{#4}{#2. Siehe auch Fu{\ss}zeile auf Seite~\pageref{fn:#1}.}}

\newcommand{\FachbegriffSpezialB}[5]{\textit{#5}\footnote{#3}\label{fn:#1}\nomenclature{#4}{#2. Siehe auch Fu{\ss}zeile auf Seite~\pageref{fn:#1}.}}


% Beträge mit Währung -----------------------------------------------------------------------------------
\newcommand{\Betrag}[2][general]{#2\,\ifthenelse{\equal{#1}{dollar}}{\$}{}\ifthenelse{\equal{#1}{euro}}{€}{}\ifthenelse{\equal{#1}{yen}}{¥}{}\ifthenelse{\equal{#1}{cent}}{¢}{}\ifthenelse{\equal{#1}{pound}}{£}{}\ifthenelse{\equal{#1}{peso}}{₱}{}\ifthenelse{\equal{#1}{baht}}{฿}{}\ifthenelse{\equal{#1}{franc}}{₣}{}\ifthenelse{\equal{#1}{lira}}{₤}{}\ifthenelse{\equal{#1}{drachma}}{₯}{}\ifthenelse{\equal{#1}{pfennig}}{₰}{}\ifthenelse{\equal{#1}{general}}{¤}{}}


% Sonstiges ---------------------------------------------------------------------------------------------
\newcommand{\Eingabe}[1]{\texttt{#1}}
\newcommand{\Code}[1]{\texttt{#1}}
\newcommand{\Datei}[1]{\texttt{#1}}

\newcommand{\Datentyp}[1]{\textsf{#1}}
\newcommand{\XMLElement}[1]{\textsf{#1}}
\newcommand{\Webservice}[1]{\textsf{#1}}


% Beschriftung von Tabellen und Bildern ändern ----------------------------------------------------------
\addto\captionsngerman{
	\renewcommand{\figurename}{Abb.}
	\renewcommand{\tablename}{Tab.}
}


% Spaltendefinition rechtsbündig mit definierter Breite -------------------------------------------------
\newcolumntype{w}[1]{>{\raggedleft\hspace{0pt}}p{#1}}


% Linksbündige Tabellenspalten mit tabularx -------------------------------------------------------------
\newcolumntype{y}[1]{>{\RaggedRight\arraybackslash\hsize=#1\hsize}X}



% Das eigentliche Dokument -----------------------------------------------------------------------------
%   Der eigentliche Inhalt des Dokuments beginnt hier. Die einzelnen Seiten
%   und Kapitel werden in eigene Dateien ausgelagert und hier nur inkludiert.
% ------------------------------------------------------------------------------------------------------
\begin{document}


% auch subsubsections nummerieren ----------------------------------------------------------------------
\setcounter{secnumdepth}{3}
% Nummerierungsebenen im Inhaltsverzeichnis
\setcounter{tocdepth}{2}


% Deckblatt und Abstract ohne Seitenzahl ---------------------------------------------------------------
\ofoot{}
% !TEX encoding = UTF-8 Unicode
% !TEX root =  Bachelorarbeit.tex

\pagenumbering{Alph}
\begin{titlepage}

\changefont{cmss}{bx}{n}

\begin{flushleft}

\LARGE{\textbf{\titel}}\\[1.5ex]
\Large{\textbf{\untertitelDeckblatt}}\\[6ex]
\Large{\textbf{\art}}\\[1.5ex]

\changefont{cmss}{m}{n}
\large{\fachgebiet \studienbereich}\\[12ex]


\includegraphics[width=0.7\textwidth]{Logo.pdf}\\[12ex]

\normalsize{}
\begin{tabular}{ll}
vorgelegt von:  & \quad \autor\\[1.2ex]
Matrikelnummer: & \quad \matrikelnr\\[1ex]
Erstgutachter:  & \quad \erstgutachter\\[1ex]
Zweitgutachter: & \quad \zweitgutachter\\[1ex]
eingereicht in: & \quad \ort, am 12.\,August\,2011
\end{tabular}

\end{flushleft}


% Die folgenden Kommentare entfernen um einen Sperrvermerk auf der 2. Seite zu erzeugen
%\singlespacing

%\newpage

%\thispagestyle{empty}

%\changefont{cmr}{m}{n}

%\small
%\noindent Die nachfolgende \art\ \textbf{enth\"alt vertrauliche Daten der Musterfabrik GmbH \& Co. Betriebs KG}. Ver\"offentlichungen oder Vervielf\"altigungen -- auch auszugsweise -- sind ohne ausdr\"uckliche Genehmigung der Musterfabrik GmbH \& Co. Betriebs KG nicht gestattet. Die \art\ ist nur den Erst- und Zweitgutachtern und dem Prüfungsausschuss zug\"anglich zu machen.\\[5ex]

\end{titlepage}

% !TEX encoding = UTF-8 Unicode
% !TEX root =  ../Bachelorarbeit.tex
\chapter*{Zitat}
\label{cha:Zitat}

\thispagestyle{empty}

\begin{center}
\begin{minipage}{14cm}
\begin{verse}
\textit{So I listen to the radio and all the songs we used to know \\
So I listen to the radio remember where we used to go}

(The Corrs -- Radio)
\end{verse}
%\hfill \textsf 
\end{minipage}
\end{center}

% !TEX encoding = UTF-8 Unicode
% !TEX root =  ../Bachelorarbeit.tex

\chapter*{Abstract}
\label{cha:Abstract}

\thispagestyle{empty}


Die vorliegende wissenschaftliche Arbeit behandelt die Planung und Entwicklung einer TYPO3 Extension zur Live-Anzeige und Nutzung von Webradio-Streams und deren Titelinformationen. Die Entwicklung soll auf Basis der am Markt neu eingef\"uhrten Frameworks Extbase und Fluid geschehen. Die Neuauflage findet im Rahmen des Relaunches der gesamten Seite des kooperierenden Radiosenders XYZ statt. 
\\
Es erfolgt eine umfangreiche Planungsphase, die in Anlehnung an den Rational Unified Process dokumentiert wird. Auf Basis dieser Planung wird die Software dann im Unternehmen selbst implementiert. Der Vorgang der praktischen Umsetzung wird schriftlich dokumentiert. Zum Abschluss der Arbeit wird die Extension im Live-Betrieb der Webseite des Unternehmens zum Einsatz kommen.
\ofoot{\pagemark \\[4ex]}


% Seitennummerierung -----------------------------------------------------------------------------------
%   Vor dem Hauptteil werden die Seiten in großen römischen Ziffern 
%   nummeriert.
% ------------------------------------------------------------------------------------------------------
\pagenumbering{Roman}
\phantomsection{} % Sorgt für korrekte Aufnahme des Inhaltsverzeichnisses in das Inhaltsverzeichnis
\addcontentsline{toc}{chapter}{Inhaltsverzeichnis}
\tableofcontents{}


% Abkürzungsverzeichnis --------------------------------------------------------------------------------
% !TEX encoding = UTF-8 Unicode
% !TEX root =  ../Bachelorarbeit.tex

\nomenclature{UKW}{Ultrakurzwelle -- Synonym f\"ur UKW-Rundfunk im Bereich 87,5 bis 108 MHz des VHF-Bandes II}
% für korrekte Überschrift in der Kopfzeile
\clearpage\markboth{\nomname}{\nomname} 
\printnomenclature{}
\label{sec:Glossar}


% arabische Seitenzahlen im Hauptteil ------------------------------------------------------------------
\clearpage{}
\pagenumbering{arabic}


% die Inhaltskapitel werden in "Inhalt.tex" inkludiert -------------------------------------------------
% !TEX encoding = UTF-8 Unicode
% !TEX root =  Bachelorarbeit.tex

% Hier können die einzelnen Kapitel inkludiert werden. Sie müssen in den 
% entsprechenden .TEX-Dateien vorliegen. Die Dateinamen können natürlich 
% angepasst werden.

% !TEX encoding = UTF-8 Unicode
% !TEX root =  ../Bachelorarbeit.tex

\chapter{Einleitung}
\label{cha:Einleitung}


\section{Das Ziel dieser Arbeit}
\label{sec:ZielDerArbeit}

Diese Bachelor-Thesis befasst sich mit der Entwicklung einer TYPO3-\FachbegriffSpezialA{Extension}{Erweiterung, Plug-in}{»An extra feature added to a standard programming language or system.« \Zitat[siehe (1)]{webopedia:extension}}{Exten\-sion} zum Anzeigen und Abspielen von Webradio-Streams des Radiosenders XYZ.

Im Zuge des Relaunches des gesamten Webauftritts des Senders war es auch n\"otig, den Webradio-Player neu zu entwickeln und direkt in das nun verwendete TYPO3-System zu integrieren.


\section{Die Umgebung, in der die Arbeit entstand}
\label{sec:EntstehungsUmgebungArbeit}

Die Entwicklung der Software geschah in Kooperation mit der Mantelgesellschaft des Radiosenders: Der Musterfabrik GmbH \& Co. Betriebs KG.

Als Entwicklungsbasis kamen die neu eingef\"uhrten Frameworks Extbase und Fluid zum Einsatz, welche derzeit noch in Entwicklung sind und deshalb nur in fr\"uhen Versionen vorliegen. Die grundlegenden Technologien, die diesen Frameworks zugrunde liegen, werden in dieser Arbeit wissentlich nicht behandelt, um den Rahmen nicht zu sprengen.


\section{Der Aufbau dieser Arbeit}
\label{sec:AufbauDieserArbeit}

\begin{description}

	\item[Aktueller Wissensstand:] Der aktuelle Wissensstand beschreibt, auf welchem Wissensniveau sich der Autor im Moment der Aufnahme der Arbeit befand.
	
	\item[Entwicklungsstand TYPO3:] Dieses Kapitel befasst sich mit dem grunds\"atzlichen Entwicklungsstand von TYPO3 Version 4 und 5 und den mit der Extension-Entwicklung zusammenh\"angenden Frameworks Extbase, Fluid und FLOW3. Es werden grundlegende Eigenschaften der Frameworks und deren Leistungsf\"ahigkeit skizziert.
	
	\item[Methoden und Herangehensweisen:] Im Kapitel »Methoden und Herangehensweisen« werden die zur Planung verwendeten Methoden erl\"autert. Die grundlegenden Eigenschaften und der Aufbau des Softwareentwicklungsprozesses \FachbegriffSpezialA{Rup}{Rational Unified Process}{»Short for Rational Unified Process, a software development methodology from Rational. Based on UML, RUP organizes the development of software into four phases, each consisting of one or more executable iterations of the software at that stage of development.« \Zitat[Abs.\,1]{webopedia:Rup}}{RUP} werden erkl\"art. Zudem werden die anzufertigenden Dokumente spezifiziert.
	
	\item[Die Planung des Webradio-Players:] Dieses Kapitel umfasst die Dokumentation der gesamten Planungsphase des Webradio-Players. Hier wird eine \"Ubersicht \"uber die bereits vorhandene L\"osung geschaffen und anschlie{\ss}end die zur Planung erforderlichen Dokumente des RUP angefertigt.
	
	\item[Die Entwicklung des Webradio-Players:] Dieses Kapitel enth\"alt die Dokumentation der tats\"achlichen Programmierung der Software. Hier werden die Voraussetzungen zur Implementation gekl\"art und der Verlauf der Entwicklung anhand von Beispielen schrittweise abgearbeitet.
	
	\item[Fazit und kritische Bewertung:] Im Fazit werden die gemachten Erfahrungen und die Ergebnisse der Planung und Entwicklung abschlie{\ss}end zusammengefasst und kritisch bewertet. Zus\"atzlich wird ein kleiner Ausblick auf Erweiterungsm\"oglichkeiten und m\"ogliche Optimierungsschritte unternommen.

\end{description}


\section{Wenige Informationen, wenige Quellen \dots}
\label{sec:Quellenlage}

Grunds\"atzlich war es schwierig geeignete Quellen zu den Themen rund um die Technologien zu finden, da sich -- wie bereits erw\"ahnt -- beide Frameworks noch in der Entwicklung befinden. Aus diesem Grund wurde \"uberwiegend aus Online-Quellen zitiert.



%\include{Inhalt/Wissensstand}

%\include{Inhalt/Entwicklungsstand}

%\include{Inhalt/Methoden}

%\include{Inhalt/Planung/Planung}
	%\include{Inhalt/Planung/Vision}
	%\include{Inhalt/Planung/Anforderungsbeschreibung}
	%\include{Inhalt/Planung/Architektur}
	
	
%\include{Inhalt/Entwicklung/Entwicklung}
	%\include{Inhalt/Entwicklung/ExtensionBuilder}
	%\include{Inhalt/Entwicklung/BeginnProgrammierung}
	%\include{Inhalt/Entwicklung/Implementierung}

%% !TEX encoding = UTF-8 Unicode
% !TEX root =  ../Bachelorarbeit.tex
\chapter{Fazit und kritische Bewertung}
\label{cha:Fazit}


\section{Das Ergebnis}
\label{sec:Ergebnis}

Der Relaunch des gesamten Webauftritts von HIT RADIO FFH erfolgte am 24.\,Juli 2011, mit dem auch der neue Webradio-Player zum ersten Produktiveinsatz gelangte. In Abbildung~\ref{fig:FfhWebradioEndversion} ist die Webradio-Player-Extension im Live-Einsatz auf \href{http://webradio.ffh.de/}{webradio.ffh.de} zu sehen. 

%% Screenshot der Endversion
\begin{figure}[htbp]
	\begin{center}
		\includegraphics[width=\textwidth]{Fazit/FfhWebradioEndversion.png}
		\caption[FFH Webradio-Player im Live-Einsatz]{Der FFH Webradio-Player im Live-Einsatz}
		\label{fig:FfhWebradioEndversion}
	\end{center}
\end{figure}

Die grunds\"atzliche Entscheidung, bei der Extension-Entwicklung auf die Frame\-works Extbase und Fluid zu setzen, hat sich als absolut richtig erwiesen. Trotz der vielen Kinderkrankheiten und der unvollst\"andigen Implementation vieler Funktionen ist die anwendungsdom\"anen-getriebene Herangehensweise von Extbase bzw. FLOW3 ein deutlicher Schritt in eine einfachere Richtung der Entwicklung. 

Der Webradio-Player l\"auft seit dem Relaunch nahezu fehlerfrei und ben\"otigte danach nur kleine Anpassungen wegen Schnittstellen\"anderungen des Stream-Providers Nacamar.


\newpage

\section{Die Bewertung der Frameworks Extbase und Fluid}
\label{sec:BewertungFrameworks}


Das bei der Radio/Tele FFH GmbH im Zuge des Relaunches zum Einsatz kommende Content Management System TYPO3 bildete die Basis der Extension-Entwicklung w\"ahrend dieser Arbeit. Aufgrund der angepriesenen Zukunftssicherheit wurde zudem entschieden, alle Extensions mit denen am Markt neu eingef\"uhrten Frameworks Extbase und Fluid zu entwickeln. Sowohl Extbase als MVC und Domain-Driven Design basiertes PHP-Framework, als auch Fluid als XML basierte Templating-Engine vereinen das gemeinsame Prinzip »Convention over Configuration«. Dieses Prinzip stellt viele Entwickler vor die gro{\ss}e Herausforderung, ihre bisherigen Gewohnheiten beim Entwickeln von Software zu \"uberdenken.

Die vielen zun\"achst \"ubertrieben erscheinenden Konventionen beim Programmieren mit Extbase und Fluid haben schlussendlich trotz vieler Zweifel erm\"oglicht, dass drei Entwickler parallel an verschiedenen Extensions arbeiten konnten, auch wenn sie nicht am urspr\"unglichen Planungsprozess beteiligt waren. Einzig durch die strengen Konventionen in der Programmierung gelingt es, sich ohne gro{\ss}e Probleme in andere Extensions einzulesen. So war es beispielsweise m\"oglich, bestimmte Quellcode-Fragmente schnell in andere Extensions zu portieren, ohne viel am bereits vorhandenen Code \"andern zu m\"ussen.

Die Templating-Engine Fluid \"ubernimmt diese hervorragenden Eigenschaften bei der Umsetzung in der View-Ebene der Extensions. Als XML-basierte Templating-Engine ist sie flexibel einsetzbar, erzeugt validen XHTML-Code und ist durch ihre Viewhelper theoretisch unendlich erweiterbar. Die bereits implementierten Viewhelper zeugen von der Macht, die von dieser Engine ausgeht. Aus diesem Grund ist es schade, dass bisher nur Basisfunktionen in Viewhelpern vorliegen. Geht es darum, exotischere Logiken in Viewhelpern zu verwenden, so ist man leider darauf angewiesen, diese selbst zu programmieren. Hierdurch geht ein nicht unerheblicher Teil an Entwicklungszeit verloren, weil die Rahmenbedingungen f\"ur die Umsetzung TYPO3 interner Funktionen leider meistens schlecht sind. Es fehlt an Dokumentationen der TYPO3 Funktionalit\"aten, aber auch an M\"oglichkeiten, diese in Viewhelpern zu realisieren. Aus diesem Grund ist man auch hier stark auf die Community im Internet angewiesen. Den Hauptanteil an Informationen bezieht man von anderen Entwicklern, die sich die L\"osungen meist umst\"andlich durch \Fachbegriff[Heuristische Methode, bei der durch Versuch und Irrtum eine L\"osung gefunden wird.]{trial \& error} erarbeitet haben.


Theoretisch l\"asst sich Fluid mit jedem verf\"ugbaren PHP-Framework verwenden. So kann eine simple Webseite auch komplett autark ohne Content Management System aufgebaut werden.


Zusammenfassend bleibt zu sagen, dass sowohl Extbase als auch Fluid ihren Weg in die Welt von TYPO3 gefunden haben. Die beiden alternativen Frameworks geben einen Vorgeschmack auf die Entwicklung mit TYPO3 v5 und FLOW3 und helfen Entwicklern schon heute sich auf die Portierung ihrer Extensions vorzubereiten. F\"ur den Autor als Neueinsteiger in der Extension-Entwicklung boten die beiden Werkzeuge viele neue und intuitive Wege, an die Entwicklung von Software heranzugehen. Nicht zuletzt ist hervorzuheben, dass die neuartigen Ans\"atze der Entwicklung, wie etwa das Domain-Driven Design und Convention over Configuration, viel Erleichterung in den Alltag eines Software-Entwicklers bringen k\"onnen, sofern dieser sich darauf einl\"asst.



\section{Ein Ausblick}
\label{sec:EinAusblick}

Es hat sich gezeigt, dass die Entwicklung der Frameworks Extbase und Fluid st\"andig voranschreitet, weshalb es in unregelm\"a{\ss}igen Abst\"anden sinnvoll ist, die Webradio-Player-Extension zu aktualisieren und Neuerungen direkt zu implementieren. Dadurch ist eine sp\"atere Umsetzung in FLOW3 f\"ur TYPO3 Version 5 einfacher zu realisieren. 

Bereits im Verlaufe der Erstellung dieser Arbeit ergaben sich \"Anderungen in Extbase, die dazu f\"uhrten, dass die Methode zum Instanziieren von Repository-Objekten ge\"andert wurde. Fr\"uher geschah dies \"uber eine \FachbegriffSpezial{Factory}{Entwurfsmuster, wirkt als Hilfsmittel zur Erzeugung von Produkten und Objekten}{»Eine Factory ist ein Hilfsmittel zur Erzeugung von Produkten und Objekten, unabhängig von konkreten Klassen. Sie wird verwendet, wenn die Instanziierung nur über den new()-Operator nicht möglich oder nicht sinnvoll ist, zum Beispiel wenn das Objekt schwierig zu konstruieren ist oder vorher konfiguriert werden muss. [\dots] Factory-Methoden sind statische Methoden zur Erzeugung von Objekten des eigenen Klassentyps. Ein Singleton kann zum Beispiel als Spezialfall einer Klasse mit einer Factory-Methode angesehen werden.« \ZitatSpezial{Vgl.}{Abschnitt „Factory“}{horn:Patterns}}{Factory}-Methode der TYPO3 Bibliothek \Code{t3lib\_div}. Mittlerweile wird ein Repository \"uber eine \FachbegriffSpezial{DependencyInjection}{Entwurfsmuster, soll helfen Abh\"angigkeiten eines Objektes einfacher aufzul\"osen}{»Dependency Injection ist ein Entwurfsmuster (Design Pattern), dass helfen soll, Abhängigkeiten eines Objektes in der objekt-orientierten Programmierung einfacher aufzulösen. Es findet eine Umkehr der Steuerung (Inversion of Control) statt, um das Objekt von unnötigen Verbindungen zu seiner Umwelt zu befreien, die es nur für die Auflösung von Abhängigkeiten, nicht aber für seine eigentliche Aufgabe benötigt. Die Verantwortung für das Auflösen der Abhängigkeiten wird aus dem Objekt in das umliegende Framework, in unserem Falle Extbase, verlagert.« \Zitat[]{oertel:DependencyInjection}}{Extbase Dependency Injection} erzeugt.

Durch die grundverschiedenen Ans\"atze beider Varianten ist schnell erkennbar, dass mit der Entwicklung Schritt gehalten werden muss, damit die eigenen Extensions zukunftssicher bleiben. Aus diesem Grund werden alle bei der Radio/Tele FFH GmbH entwickelten Extensions -- so auch der Webradio-Player -- st\"andig auf dem neusten Entwicklungsstand gehalten. 

Einer Portierung auf FLOW3 st\"unde somit nichts im Wege.








\clearpage{}
\pagenumbering{Roman}
\setcounter{page}{7} %%% Dieser Pagecounter muss entsprechend der verbrauchten Seiten im Inhaltsverzeichnis angepasst werden. Endet das IHV bei Seite III, so muss hier 4 eingetragen werden


% Literaturverzeichnis ---------------------------------------------------------------------------------
%   Das Literaturverzeichnis wird aus der BibTeX-Datenbank "Bibliographie.bib"
%   erstellt.
% ------------------------------------------------------------------------------------------------------
\bibliography{Bibliographie} % Aufruf: bibtex Masterarbeit
\bibliographystyle{natdin} % DIN-Stil des Literaturverzeichnisses


% Restliche Verzeichnisse ------------------------------------------------------------------------------
\listoffigures{} % Abbildungsverzeichnis
\listoftables{} % Tabellenverzeichnis
\renewcommand{\lstlistlistingname}{Verzeichnis der Listings}
\lstlistoflistings{} % Listings-Verzeichnis


% Index ------------------------------------------------------------------------------------------------
%   Zum Erstellen eines Index, die folgende Zeile auskommentieren.
% ------------------------------------------------------------------------------------------------------
%\printindex


% Selbständigkeitserklärung ----------------------------------------------------------------------------
% !TEX encoding = UTF-8 Unicode
% !TEX root =  Bachelorarbeit.tex

\addchap{Eidesstattliche Versicherung}
Ich, \autor, Matrikel-Nr.\ \matrikelnr, versichere hiermit, dass ich meine \art\xspace mit dem Thema
\begin{quote}
\textit{\titel} \textit{\untertitel}
\end{quote}
selbständig verfasst und keine anderen als die angegebenen Quellen und Hilfsmittel benutzt habe, wobei ich alle wörtlichen und sinngemäßen Zitate als solche gekennzeichnet habe. Die Arbeit wurde bisher keiner anderen Prüfungsbehörde vorgelegt und auch nicht veröffentlicht.

Mir ist bekannt, dass ich meine \art\xspace zusammen mit dieser Erklärung fristgemäß nach Vergabe des Themas in dreifacher Ausfertigung und gebunden im Prüfungsamt der \hochschule\xspace abzugeben oder spätestens mit dem Poststempel des Tages, an dem die Frist abläuft, zu senden habe.\\[6ex]

\ort, den \today


\rule[-0.2cm]{5cm}{0.5pt}

\textsc{\autor} 
 


% Anhang -----------------------------------------------------------------------------------------------
%   Die Inhalte des Anhangs werden analog zu den Kapiteln inkludiert.
%   Dies geschieht in der Datei "Anhang.tex".
% ------------------------------------------------------------------------------------------------------
\begin{appendix}
    \clearpage{}
    \pagenumbering{roman}
    \chapter{Anhang}
    \label{sec:Anhang}
    % Rand der Aufzählungen in Tabellen anpassen
    \setdefaultleftmargin{1em}{}{}{}{}{}
    % !TEX encoding = UTF-8 Unicode
% !TEX root =  Bachelorarbeit.tex

\section{Beiliegende CD}
\label{sec:BeiliegendeCd}

\subsection{Inhaltsverzeichnis der CD}
\label{subsec:InhaltsverzeichnisCd}

\begin{enumerate}
	\item Die gesamte Bachelorarbeit als PDF-Datei
	\item Alle verwendeten Online-Quellen als PDF-Ausdruck
	\item Sonstige Quelltexte
	\item Alle in der Bachelorarbeit verarbeiteten Diagramme
	\item Alle in der Bachelorarbeit verarbeiteten Grafiken
	\item Alle in der Bachelorarbeit verarbeiteten Screenshots
\end{enumerate}
\end{appendix}

\end{document}
